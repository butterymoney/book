% Options for packages loaded elsewhere
\PassOptionsToPackage{unicode}{hyperref}
\PassOptionsToPackage{hyphens}{url}
%
\documentclass[
]{article}
\usepackage{amsmath,amssymb}
\usepackage{lmodern}
\usepackage{iftex}
\usepackage{graphicx}
\ifPDFTeX
  \usepackage[T1]{fontenc}
  \usepackage[utf8]{inputenc}
  \usepackage{textcomp} % provide euro and other symbols
\else % if luatex or xetex
  \usepackage{unicode-math}
  \defaultfontfeatures{Scale=MatchLowercase}
  \defaultfontfeatures[\rmfamily]{Ligatures=TeX,Scale=1}
\fi
% Use upquote if available, for straight quotes in verbatim environments
\IfFileExists{upquote.sty}{\usepackage{upquote}}{}
\IfFileExists{microtype.sty}{% use microtype if available
  \usepackage[]{microtype}
  \UseMicrotypeSet[protrusion]{basicmath} % disable protrusion for tt fonts
}{}
\makeatletter
\@ifundefined{KOMAClassName}{% if non-KOMA class
  \IfFileExists{parskip.sty}{%
    \usepackage{parskip}
  }{% else
    \setlength{\parindent}{0pt}
    \setlength{\parskip}{6pt plus 2pt minus 1pt}}
}{% if KOMA class
  \KOMAoptions{parskip=half}}
\makeatother
\usepackage{xcolor}
\IfFileExists{xurl.sty}{\usepackage{xurl}}{} % add URL line breaks if available
\IfFileExists{bookmark.sty}{\usepackage{bookmark}}{\usepackage{hyperref}}
\hypersetup{
  hidelinks,
  pdfcreator={LaTeX via pandoc}}
\urlstyle{same} % disable monospaced font for URLs
\setlength{\emergencystretch}{3em} % prevent overfull lines
\providecommand{\tightlist}{%
  \setlength{\itemsep}{0pt}\setlength{\parskip}{0pt}}
% \setcounter{secnumdepth}{-\maxdimen} % remove section numbering
\ifLuaTeX
  \usepackage{selnolig}  % disable illegal ligatures
\fi

\title{Corruption is not enough}
\author{Butter}
\date{December 2022}

\begin{document}
\maketitle
\setcounter{tocdepth}{2}
\tableofcontents

\hypertarget{butter}{%
\section{Butter}\label{butter}}

Butter is a DAO governance system. Its objective is to align DAO
governance outcomes to DAO objectives through the direct application of
incentives.

This paper describes a prototype, Molten, that attempts to influence DAO
stakeholders and governance participants to cooperate using one-off
payments.

\hypertarget{summary}{%
\section{Summary}\label{summary}}

One token, one vote mechanisms are the most popular governance
mechanisms used by DAOs. However, one token, one vote, and other token
voting systems are plutocracies and widely considered vulnerable to
\href{./problems.md\#corruption-problems}{corruption} and
\href{./problems.md\#attack-problems}{attack}.

Large or mature DAOs, and many newer DAOs, have introduced vote
delegation to scale governance and alleviate the risks of plutocracy.

However, many recent and well-known examples of plutocracy involve DAOs,
including ENS DAO, Sushi DAO, and MakerDAO, in which vote delegation is
allowed and often enforced.

We examine the DAO governance problem space and highlight promising
in-market solutions, including hybrid governance, metagovernance, and
market-based governance.

We propose cryptoeconomics, a system of economic incentives designed to
produce behaviors at the micro-scale that create desirable, emergent
properties at the macro scale, as a viable solution space for further
exploration.

Finally, we describe Molten, a system that aims to address the risks
related to plutocracy in DAO Governance using incentives designed to
make corruption an unprofitable strategy.
\hypertarget{motivation}{%
\section{Motivation}\label{motivation}}

\hypertarget{decentralised-autonomous-organisations}{%
\subsection{Decentralised Autonomous
Organisations}\label{decentralised-autonomous-organisations}}

DAOs are a novel form of organization uniquely enabled by blockchains.

The components of an organization include, but are not limited to:

\begin{enumerate}
\def\labelenumi{\arabic{enumi}.}
\tightlist
\item
  an \textbf{objective} or \textbf{purpose}
\item
  a \textbf{membership policy} that produces a set of members, e.g.,
  holds shares, holds tokens, is contracted.
\item
  an \textbf{allocation mechanism} which defines how the organization
  allocates resources
\item
  a \textbf{standardized store of value} which defines how the
  organization values and represents its resources, e.g., currency,
  equity, tokens
\item
  a \textbf{governance mechanism} that defines how to update the
  organization's properties, e.g., membership policies and allocation
  mechanisms.
\end{enumerate}

\begin{figure}
\centering
\includegraphics[width=\textwidth]{./img/components.png}
\caption{DAO Components}
\end{figure}

DAOs, through their use of blockchains, claim to provide the benefits of
large-scale coordination without the downsides of centralization. These
downsides include capture, corruption, and collusion----problems that
undermine our most-trusted institutions.

Generally, DAO proponents expect DAOs to replace traditional
institutions in providing public, common, or club goods. When these
institutions fail, centralization is usually a root cause, e.g.,
bureaucracy, corruption, and principal-agent problems.

In practice, however, DAOs do not offer solutions to these problems and
often move them
\href{https://kelsienabben.substack.com/p/towards-a-model-of-resilience-in}{elsewhere
in the value chain}. Current DAO implementations, therefore, remain
vulnerable to the same issues faced by our incumbent institutions.

\hypertarget{dao-governance}{%
\subsection{DAO Governance}\label{dao-governance}}

\emph{DAO governance involves a network of participants that coordinate
to make decisions, \textbf{without a centralized actor with privileged
rights,} in pursuit of some goal or outcome, and is formalized or
defined under a set of shared context(s), e.g., a territory, the law, a
market, or a cause.}

DAOs, like other organizations, implement internal policies that govern
their components and the interactions between them. Internal policies
include the law in the case of nations, compensation, taxation, resource
allocation, and social choice.

Similarly, external policies enforced by the environment govern
organizations. External policies include corporate law, market forces,
international relations, physics, and blockchain protocols.

Therefore, governance mechanisms can be considered the component of DAOs
responsible for mediating all DAO components. They are, in turn,
mediated by their environment and competing DAOs.

\begin{figure}
\centering
\includegraphics[width=\textwidth]{./img/governance_influence.png}
\caption{Governance Influence}
\end{figure}

Considering a DAOs ability to affect its outcomes, a DAO's governance
mechanism can be considered the DAO itself. Therefore, we expect
improvements in DAO Governance to be an effective means to realize the
expected positive value of DAOs on society.

\hypertarget{dao-governance-models}{%
\subsection{DAO Governance Models}\label{dao-governance-models}}

\emph{Note: We recognize that token voting, though democratic in nature,
is far from a democracy in the literal sense. However, we will use the
term democracy to adhere to market convention}

Models include:

\begin{itemize}
\tightlist
\item
  Direct Democracy
\item
  Representative Democracy
\item
  Reputation-based Voting
\end{itemize}

\hypertarget{direct-democracy}{%
\subsubsection{Direct Democracy}\label{direct-democracy}}

\textbf{\emph{One token, one vote on every proposal}}

\textbf{Description}

In a direct democracy, token-holders make decisions by voting on
proposals, where each token is equivalent to a vote. Direct Democracy is
the governance mechanism used by most DAOs, specifically smaller,
younger DAOs.

Governance must configure the following parameters:

\begin{itemize}
\tightlist
\item
  Who has the right to create a proposal
\item
  How to convert token votes to a decision, e.g., majority-rule,
  supermajority, quorum rules
\end{itemize}

\textbf{Benefits}

\begin{itemize}
\tightlist
\item
  Bundling financial upside and governance rights aligns risk and
  responsibility, which incentivizes those with the most to gain from
  price appreciation to make decisions that directly or indirectly
  maximize price appreciation
\item
  This replicates features of the equity system, which makes it simple
  for holders to understand
\end{itemize}

\textbf{Limitations}

\begin{itemize}
\tightlist
\item
  Keeps out those who may be affected by governance but do not have the
  capital to acquire governance rights
\item
  Tends towards plutocracy, which, if left unchecked, leads to failure
  through a focus on price appreciation, regardless of negative
  externalities
\end{itemize}

\textbf{Examples}

\begin{itemize}
\tightlist
\item
  PleasrDAO, Aavegotchi, VitaDAO
\end{itemize}

\hypertarget{representative-democracy}{%
\subsubsection{Representative
Democracy}\label{representative-democracy}}

\textbf{\emph{One token, one vote on every proposal with vote
delegation}}

\textbf{Description}

In a representative democracy, token-holders make decisions by voting on
proposals, where each token is equivalent to a vote but can also
delegate their voting power to a representative. Delegated voting is
increasingly becoming the most popular governance mechanism, especially
for mature, large DAOs.

Governance must configure the following parameters:

\begin{itemize}
\tightlist
\item
  Who has the right to create a proposal
\item
  How to convert token votes to a decision, e.g., majority-rule,
  supermajority, quorum rules
\item
  Which rights are delegatable and to whom
\end{itemize}

\textbf{Benefits}

\begin{itemize}
\tightlist
\item
  Aligns incentives by unbundling financial risk and governance power
  and allocating them to domain experts
\item
  Allows governance rights to accrue to representatives whom voters
  believe best represent their preferences
\item
  Reduces voter apathy
\end{itemize}

\textbf{Limitations}

\begin{itemize}
\tightlist
\item
  As delegation scales, the nuance of voter preferences is diluted to
  the preferences of a smaller subset of voters, i.e., the delegates,
  which is less representative of the population
\item
  Forces the voter to find a single delegate who represents their entire
  range of preferences across all possible decisions (though voters can
  split tokens across wallets or enhance delegation functionality)
\item
  Allowing voters to delegate enables a more persistent form of voter
  apathy, as seen in our traditional political system
\end{itemize}

\textbf{Examples}

\begin{itemize}
\tightlist
\item
  Uniswap, Gitcoin, Compound, ENS, MakerDAO, AAVE, Radicle, Nouns DAO
\end{itemize}

\hypertarget{reputation-based-voting}{%
\subsubsection{Reputation-based Voting}\label{reputation-based-voting}}

\textbf{\emph{One person, one vote OR One contribution/reputation unit,
one vote on every proposal}}

\textbf{Description}

Non-transferable voting. Based on an entity's membership, reputation,
and, or contribution.

\textbf{Benefits}

\begin{itemize}
\tightlist
\item
  more equitable relative to token-weighted voting, i.e., meritocratic
\item
  aligns contribution and power
\item
  does not produce plutocracy
\end{itemize}

\textbf{Limitations}

\begin{itemize}
\tightlist
\item
  only as performant as the system's ability to measure contributions
  and assign relative value
\item
  assumes equal exposure to externalities
\item
  inability to express preference intensity
\end{itemize}

\textbf{Examples}

\begin{itemize}
\tightlist
\item
  Optimism's Citizen's House
\end{itemize}
\hypertarget{problems}{%
\section{Problems}\label{problems}}

\hypertarget{problem-space}{%
\subsection{Problem Space}\label{problem-space}}

The problem space is DAO Governance, in particular:

\begin{itemize}
\tightlist
\item
  DAO Governance Corruption, including Capture, Collusion, and
  Opportunism
\item
  DAO Governance Attacks, including Capital Structure Exploitation
\end{itemize}

\hypertarget{properties}{%
\subsection{Properties}\label{properties}}

\begin{itemize}
\tightlist
\item
  \textbf{Stakeholder.} Any individual, collective, or entity that
  experiences externalities due to the actions of the DAO, e.g.,
  Token-holder, user, delegate, staker/miner.
\item
  \textbf{Participant.} Any individual, collective, or entity that
  participates in governance
\item
  \textbf{Preference.} A stakeholder's subjective, comparative
  evaluations over a range of options, e.g., a miner prefers to increase
  the block reward over reducing rewards or keeping rewards constant
\item
  \textbf{Objectives.} The goal or set of goals that constitute the
  DAO's organizing purpose, e.g., ``Buy the constitution,'' ``Fund
  Public Goods''
\item
  \textbf{Acts.} The set of actions or decisions the DAO's governance
  mechanism can produce and its stakeholders consider, e.g., Add a new
  asset as collateral in our lending protocol, remove a particular
  voter's voting power, increase token supply, offboard a contributor,
  suspend the protocol
\item
  \textbf{Outcomes.} The set of outcomes the DAO's governance mechanism
  can achieve through its actions, e.g., Token Price increases or
  remains stable, protocol users increase
\end{itemize}

\hypertarget{dimensions}{%
\subsection{Dimensions}\label{dimensions}}

To measure the effectiveness of a DAO's governance, we consider the
following dimensions:

\begin{itemize}
\tightlist
\item
  \textbf{Stakeholder Representation.} The distribution of voting power
  relative to DAO stakeholders, i.e., users, token holders, stakers, and
  liquidity providers.
\item
  \textbf{Preference Representation.} The degree to which governance
  participants can express their preferences concerning the DAO's
  objectives, e.g., a voter does not believe the voting mechanism is
  legitimate
\item
  \textbf{Alignment.} The consistency of a decision when compared to a
  desired outcome
\item
  \textbf{Coherence.} The consistency of a series of decisions when
  compared to one another concerning a desired outcome
\item
  \textbf{Legitimacy.} Power granted by governance participants to the
  governance mechanism through their ongoing implicit agreement to be
  bound by its decisions
\end{itemize}

\hypertarget{problems-1}{%
\subsection{Problems}\label{problems-1}}

\hypertarget{corruption-problems}{%
\subsubsection{Corruption Problems}\label{corruption-problems}}

\hypertarget{opportunism}{%
\paragraph{Opportunism}\label{opportunism}}

Where a single stakeholder or group of stakeholders receives rewards for
acting in self-interest while punishing all other stakeholders and
producing outcomes that do not align with the DAO's objectives.

\textbf{Example:} Proposing or voting for salary increases or against
salary cuts during a budget-cutting exercise.

\textbf{Symptoms:} - Deviation between outcomes and objectives -
Increase in actions or decisions that do not align with objectives -
Illegitimate diversion of funds

\hypertarget{capture}{%
\paragraph{Capture}\label{capture}}

Where a minority group of stakeholders possesses the power to dictate
the DAO's actions to serve their self-interest while punishing all other
stakeholders and producing outcomes that do not align with the DAO's
objectives.

\textbf{Example:} Plutocracy, Bureaucracy

\textbf{Symptoms:} - Deviation between outcomes and objectives -
Increase in actions or decisions that do not align with objectives -
Illegitimate diversion of funds

\hypertarget{collusion}{%
\paragraph{Collusion}\label{collusion}}

Where two or more stakeholders or stakeholder groups operating within or
outside the boundaries of the DAO cooperate for their mutual benefit, to
the detriment of all other stakeholders and the DAO's ability to achieve
its objectives.

\textbf{Example:}
\href{https://hackingdistributed.com/2018/07/02/on-chain-vote-buying/}{Vote
Buying}

\textbf{Symptoms:} - Deviation between outcomes and objectives -
Increase in actions or decisions that do not align with objectives -
Illegitimate diversion of funds

\hypertarget{attack-problems}{%
\subsubsection{Attack Problems}\label{attack-problems}}

\hypertarget{capital-structure-exploitation}{%
\paragraph{Capital Structure
Exploitation}\label{capital-structure-exploitation}}

Where an individual or group can exploit vulnerabilities in the DAO's
governance mechanism to extract capital.

\textbf{Example:} Treasury Drain Attacks, Price Manipulation Attacks,
and Arbitrageurs.

\textbf{Symptom:} - Illegitimate diversion of funds
\hypertarget{governance-innovations}{%
\section{Governance Innovations}\label{governance-innovations}}

Unlike corporate and public governance, DAO Governance is both public
and open source. The principles of the open-source community, which
anyone can copy and reuse for free, provide many opportunities for
governance innovation----some of which we have shared below.

\emph{Metagovernance,} \emph{Hybrid Governance,} and \emph{Market
Governance} are three categories of governance innovation that may offer
practical solutions concerning our problem space.

\hypertarget{metagovernance}{%
\subsection{Metagovernance}\label{metagovernance}}

Metagovernance, in the context of DAOs, is the term commonly used to
describe any activity where one governance mechanism, typically a
protocol or a DAO, exerts influence on the governance of another DAO.

Metagovernance is a transparent, often automated, vote-buying mechanism
that incentivizes a target DAO's token-holders to take an action that
benefits the mechanism's stakeholders, e.g., influence over governance
decisions and the direction of token emissions.

Metagovernance creates a secondary set of incentives, or
meta-incentives, that augment the behavior of the target DAO's
stakeholders.

In one-off instances of metagovernance, such as in the case of Fei and
Index Coop, the Fei team gained influence in AAVE's governance using
Index Coop's AAVE holdings.

There are also extended forms of metagovernance with DAOs whose entire
purpose is to control the governance of other DAOs, such as Convex
Finance.

\hypertarget{curve-emissions-with-convex-finance}{%
\subsubsection{Curve emissions with Convex
Finance}\label{curve-emissions-with-convex-finance}}

Convex maximizes control over CRV emissions on the Curve protocol.

Convex works by reimplementing Curve's vote-escrow token mechanic to pay
CRV holders with CVX emissions in exchange for locking their CRV tokens
in Convex's contract.

Convex, in turn, locks these CRV tokens using Curve's contracts to
maximize their CRV emissions, which they share with CVX holders, and
voting power, which they use to vote for increased token emissions for
pools selected by CVX holders.

As of this writing, the Convex protocol controls 51\% of all
vote-escrowed CRV, an indicator of the effectiveness of meta-incentives
in one-token, one-vote governance mechanisms.

\hypertarget{redacted-cartel-hidden-hand}{%
\subsubsection{Redacted Cartel, Hidden
Hand}\label{redacted-cartel-hidden-hand}}

Hidden Hand from Redacted Cartel facilitates vote-buying campaigns for
participating DAOs.

Vote buyers, or bribers, can deposit bribes for governance proposals at
participating DAOs, and users can delegate governance tokens to the
Hidden Hand protocol. The protocol then distributes votes to maximize
returns for its users in exchange for a 4\% commission of bribes
received.

For example, as of September 7, 2022, \$851,364 worth of bribes were
deposited on Aura Finance and \$2,346,024 on Balancer.

Hidden Hand also allows its partners to operate bribe marketplaces so
their token holders can sell votes to bribers directly.

\hypertarget{fei-asset-listing-on-aave-with-index-coop}{%
\subsubsection{FEI Asset Listing on AAVE with Index
Coop}\label{fei-asset-listing-on-aave-with-index-coop}}

Index Coop, a provider of token indexes, actively encouraged
metagovernance for a small number of the tokens held in their DeFi Pulse
Index, namely Maker, AAVE, and Compound--a service they promoted as
metagovernance-as-a-service.

Under this arrangement, holders of INDEX tokens could use governance
tokens held as part of the index service to make or vote on proposals
within MakerDAO, AAVE, and Compound.

In September 2021, Fei protocol, a stablecoin issuer, created a proposal
to list the FEI token on AAVE, using the AAVE token holdings in Index
Coop's DPI.

AAVE's governance requires 80,000 AAVE tokens before a holder can make a
governance proposal. At that time, AAVE was trading at \$327.04, setting
the cost of proposal creation on AAVE at over \$26m.

As a result of metagovernance, the Fei team used \$4m of INDEX tokens to
control over 118,000 AAVE, worth \textasciitilde\$36m, allowing the team
to list their token on AAVE successfully.

\hypertarget{hybrid-governance}{%
\subsection{Hybrid Governance}\label{hybrid-governance}}

Hybrid governance combines two or more governance models within a single
DAO governance mechanism.

DAOs typically pursue this approach where DAO governance designers
believe they can better align outcomes to the DAO's objectives by
limiting the influence of stakeholders whose preferences are
over-represented in a one-token, one-vote model.

Implementing hybrid governance can also give greater weight to the
preferences of a group of stakeholders who are underrepresented or have
no means to express their preferences except to ``vote with their
feet,'' which is a loss for all stakeholders.

Hybrid governance modulates the influence of one set of stakeholders by
distributing voting rights to another set of stakeholders, especially
groups that may be marginalized by the preferences of dominant voters.

Voting power is redistributed until each group can provide sufficient
checks and balances on the power of other groups.

\hypertarget{lidos-steth-dual-governance}{%
\subsubsection{Lido's stETH Dual
Governance}\label{lidos-steth-dual-governance}}

LidoDAO's is governed by LDO holders. Unfortunately, users that stake
ETH in the Lido contract receive stETH, which confers no voting rights
to the holder.

This structure allows Lido holders to make decisions that benefit LDO
holders at the expense of stETH holders.

The goal of Lido's Dual Governance proposal ``is to prevent the Lido DAO
governance from changing the covenant between the protocol and stakers
without consent from the latter.''

The proposal grants stETH holders a vetocracy over proposals deemed to
break the agreement under which users stake their ETH on Lido. stETH
holders can signal their disagreement with a proposal by staking stETH
in a vote escrow contract. Once staked stETH reaches a threshold, the
proposal is temporarily blocked to allow the community to negotiate.
stETH holders can vote to block, amend, or pass the proposal after
negotiations.

This power gives stETH enough power to limit opportunism by LDO token
holders without burdening stakers with ongoing governance overhead.

\hypertarget{optimisms-hybrid-governance}{%
\subsubsection{Optimism's Hybrid
Governance}\label{optimisms-hybrid-governance}}

Optimism, through the Optimism Collective, has implemented a bicameral
legislative process comprising a `Token House' which grants voting
powers through token ownership, and a `Citizens' House,' which grants
voting powers through non-transferrable NFTs or ``soulbound tokens.''

The Citizen's House reserves its remit for retroactive public goods
funding. In contrast, the Token House's remit is closer to traditional
DAO governance, e.g., governance fund grants, protocol upgrades,
director removal, and so on. The team explains that this approach is ``a
large-scale experiment in
\href{https://vitalik.ca/general/2021/08/16/voting3.html}{non-plutocratic
governance},'' but so far, there are few details.

\hypertarget{market-governance}{%
\subsection{Market Governance}\label{market-governance}}

Market Governance is co-opted from
\href{https://en.wikipedia.org/wiki/Market_governance_mechanism}{Market
Governance Mechanisms} to describe a mechanism that leverages the
competitive forces of the open market to influence the behavior of
stakeholders.

As DAOs have scaled in scope, market cap, and contributors, token-based
governance has created opportunities for corruption and in-fighting,
especially where the DAO's operations are complex.

As the range and diversity of stakeholders increases and the potential
set of actions and decisions expand, governance must increase its
throughput to accommodate without creating a self-serving bureaucracy.

\hypertarget{makerdaos-metadaos}{%
\subsubsection{MakerDAO's MetaDAOs}\label{makerdaos-metadaos}}

The solution proposed by Rune in Endgame constitutes a decomposition of
MakerDAO into a single core DAO housing the main functions of the Maker
protocol and a collection of smaller ``MetaDAO'' governance units with
their own governance and governance token. ``MetaDAOs'' can pursue any
viable market opportunity while leveraging the resources of the core
DAO.

This innovation allows MakerDAO to maintain several
governance-controlled parameters for the core protocol. At the same
time, the market provides the incentives to steer MetaDAO governance to
pursue new market opportunities.

MakerDAO is in the process of deploying this governance upgrade, so its
effects are yet to be measured.
\hypertarget{proposal}{%
\section{Proposal}\label{proposal}}

Each governance innovation provides a tool for addressing problems in
DAO Governance.

\emph{Metagovernance} provides a system of secondary incentives to
reward governance participants for taking a set of desired actions,
which helps limit opportunistic behavior.

\emph{Hybrid Governance} redistributes voting power among stakeholders
using a secondary governance mechanism. The second mechanism provides a
system of checks and balances between the two groups, guaranteeing that
a single group of voters can only capture a governance mechanism
partially.

\emph{Market Governance} decomposes governance into self-contained
organizations, allowing market forces to govern each organization's
decisions. As a result, consistently productive organizations outcompete
organizations that become bureaucracies or engage in corruption.

These mechanisms improve stakeholder cooperation and alignment using
three types of incentives. \emph{Metagovernance} incentives are
incentives provided by a mechanism or protocol. \emph{Hybrid Governance}
relies on incentives provided by other stakeholders or peers.
\emph{Market Governance} incentives are incentives provided by the
market through competition.

Should governance attack, corruption, or capture be detectable by a
governance mechanism, other stakeholders, or competing mechanisms,
incentives offer an opportunity to offset the proceeds of these acts,
rendering them unprofitable.

Through our investigations in the field of cryptoeconomics, we aim to
design a mechanism or collection of mechanisms capable of providing
sufficient guarantees about cooperative behavior in DAOs.

Next, we present an initial exploration of the solution space.
\hypertarget{molten}{%
\section{Molten}\label{molten}}

\emph{Molten is WIP. This document will be updated as we conduct ongoing
research and development.}

Molten incentivizes voters to pool their votes to deter opportunistic
behavior by large token holders.

\hypertarget{actors}{%
\subsection{Actors}\label{actors}}

Molten coordinates the actions of three actors:

\textbf{Representative.} Stakeholder that wants to accrue voting power
but needs more resources. Equivalent to a delegate in a DAO with vote
delegation.

\textbf{Voter.} Stakeholder that wants protection against large voters
and has adequate resources.

\textbf{Target DAO.} DAO that wants to allocate resources in pursuit of
its objective.

\hypertarget{components}{%
\subsection{Components}\label{components}}

Molten's components include:

\textbf{Campaigns.} A period within which Voters pool their tokens in
order to delegate voting power to a Representative. Can be launched by
the Representative once deposits exceed the Campaign threshold, e.g.,
100,000 governance tokens deposited.

\textbf{Campaign Manager.} Contract that manages all Campaigns,
including types, parameters, and rewards.

\textbf{Molten Pot.} Contract created by Representatives to lock tokens
during a Campaign. Voters deposit and claim tokens from Pots. Launching
the Campaign freezes deposits.

\textbf{Molten Pot Factory.} Contract used to create Pots.

\textbf{Reward.} Fixed token reward, e.g., 30,000 DAI, claimable in
proportion to Voters' Pot claim at the end of a Campaign.

\textbf{mTokens.} Tokens issued to Voters during a Campaign. Each token
represents a proportional claim on the underlying governance tokens and
rewards locked in the Pot contract.

\hypertarget{operation}{%
\subsection{Operation}\label{operation}}

\begin{figure}
\centering
\includegraphics[width=\textwidth]{./img/molten_interactions.png}
\caption{Components Interacting}
\end{figure}

\begin{enumerate}
\def\labelenumi{\arabic{enumi}.}
\item
  Molten uses the Campaign Manager to create a new Campaign Type and
  sets Campaign duration, Target DAO Governance Token (ERC20)
\item
  Representatives create Campaigns and broadcast their Campaign and
  Molten Pot contract address to Voters
\item
  Voters deposit tokens into Molten Pots attached to Representatives
  they believe will pool the most voting power for a Campaign
\item
  Target DAO deposits Reward in the Campaign Manager for the respective
  Campaign Type and sets Campaign threshold
\item
  Once a Molten Pot attached to a Representative exceeds the Campaign
  threshold, the Representative can launch the Campaign. Campaign
  launches end all other Campaigns of the same type, allowing Voters to
  claim tokens from the Molten Pots
\item
  At the Campaign launch, Molten Pots will delegate pooled governance
  token voting power to Representatives, e.g., AAVE, and issue mTokens
  to Voters, e.g., mAAVE
\item
  After a Campaign reaches duration, the underlying governance tokens
  and Rewards are claimable from the Molten Pot \& Campaign Manager
\item
  During the Campaign, should mToken holders decide that a
  Representative is no longer protecting their interests, they can vote
  to terminate the Campaign, removing the Representative's voting power
  and forfeiting the Reward for all parties
\end{enumerate}

\hypertarget{outcome}{%
\subsection{Outcome}\label{outcome}}

Molten combines peer incentives and market incentives. It uses peer
incentives to incentivize Representatives to identify governance capture
and corruption and increase stakeholder cooperation. It uses market
incentives to surface and distribute voting power to the most competent
and motivated Representatives capable of keeping the most influential
voters in check.

Peer incentives will provide sufficient rewards to Representatives that
identify corruption or
capture\href{https://doi.org/10.1371/journal.pcbi.1004232}{2} in the
form of delegated voting power to limit potential gains from either
strategy, creating an effective deterrent.

Market incentives should encourage competent Representatives to emerge,
able to monitor and counter opportunism by large voters.

Given the presence of both peer and market incentives, coordination
costs are the most significant barrier to voters pursuing a collective
voting strategy. We aim to prove the effectiveness of incentives on
voter coordination.

\hypertarget{implementation}{%
\subsection{Implementation}\label{implementation}}

See prototype implementation
\href{https://github.com/butterymoney/molten/}{here}.

\end{document}
